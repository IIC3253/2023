%!TEX root = ../main/main.tex
%

Al usar firmas de Schnorr, suponemos dados un grupo $G$, un elemento $g$ de dicho grupo, y un número $q$ tal que el orden del subgupo generado por $g$ es $q$ (es decir, $|\langle g\rangle|=q$).

Adicionalmente se pide que $q$ sea un número primo. Explique por qué.
\paragraph{Respuesta:} Se pide que $q$ sea un número primo para que cualquier persona pueda verificar que $q$ es efectivamente el orden del subgrupo generado por $g$. Para esto basta verificar que $g^q=e$ (el elemento neutro de $G$) y que $q$ es efectivamente un número primo.

% Suponga que necesita tener una base de datos con información que permita autentificar usuarios en base a un password que los mismos usuarios proveen (como suele ocurrir en la Web). Explique
% %%%%%en no más de cinco líneas
% por qué \textbf{no} sería una buena idea almacenar pares $(e,H(p))$, donde $e$ es el correo del usuario, $H$ es una función de hash criptográfica y $p$ es el password del usuario. \textit{Hint: Comience su respuesta con ``Si se filtra la base de datos [$\ldots$]''}
