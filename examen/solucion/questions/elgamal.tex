%!TEX root = ../main/main.tex

En el protocolo de ElGamal, suponemos dados un grupo $G$, un elemento $g$ de dicho grupo, y un número $q$ tal que el orden del subgupo generado por $g$ es $q$ (es decir, $|\langle g\rangle|=q$).

Adicionalmente se pide que $q$ sea un número {\it grande}. Explique por qué.
\paragraph{Respuesta:} Se pide que $q$ sea un número {\it grande} porque si no lo fuera entonces el problema del logaritmo discreto en $\langle g\rangle$ se podría resolver por fuerza bruta.

% Suponga que encuentra en la calle un papel que en un lado tiene un par ordenado de números y en el otro dice literalmente ``El par ordenado al reverso representa un número que ha sido encriptado con ElGamal sobre $\mathbb{Z}_p^*$, donde $p$ vale 55340232221128654847''.
%
% ¿Es la información disponible suficiente para decriptar el número original? De no ser así: ¿Qué información faltaría?
