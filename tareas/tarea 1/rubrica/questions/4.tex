%!TEX root = ../main/main.tex

En ayudantía fue demostrado que si una función de hash es resistente a
colisiones, entonces esta función debe ser resistente a preimagen. En
esta pregunta usted debe demostrar que la implicación inversa no es
cierta. Vale decir, suponiendo que existe una función de hash que es
resistente a preimagen, demuestre que existe una función de hash
$(\Gen, h)$ que es resistente a preimagen y no es resistente
a colisiones.

\medskip

\paragraph{Corrección.}
La asignación de puntaje en esta pregunta es la siguiente.
\begin{itemize}
    \item{[1.5 puntos]} Sólo se entrega la definición de la función de
    hash $(\Gen, h)$, la cual no está completamente correcta.

\item{[3 puntos]} Se entrega una definición correcta de la función de
    hash $(\Gen, h)$.

    \item{[4.5 puntos]} Se entrega una definición correcta de la función de
    hash $(\Gen, h)$, y se demuestra que es resistente a preimagen. 

    \item{[6 puntos]} Se entrega una definición correcta de la función de
    $(\Gen, h)$, se demuestra que es resistente a preimagen y se
    demuestra que no es resistente a colisiones.
\end{itemize}

\medskip
