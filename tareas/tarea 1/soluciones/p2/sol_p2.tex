\documentclass[11pt]{article}


\usepackage[utf8]{inputenc}
\usepackage{fullpage}
\usepackage{epsfig}
\usepackage{amsmath}
\usepackage{amssymb}
\usepackage{multicol}
\usepackage{color}
\usepackage{hyperref}
\usepackage{xcolor}
\usepackage{dirtree}
\usepackage{fontawesome}
\usepackage{tikz}


\usetikzlibrary{trees}


\newcommand{\comm}[1]{{\bf {\color{red} #1}}}
\newcommand{\Enc}{\textit{Enc}}
\newcommand{\Dec}{\textit{Dec}}
\newcommand{\Gen}{\textit{Gen}}

\newcommand{\M}{\mathcal{M}}
\newcommand{\C}{\mathcal{C}}
\newcommand{\K}{\mathcal{K}}

\begin{document}

\begin{center}
  \bf Criptografía y Seguridad Computacional - IIC3253\\
  \bf Tarea 1\\
  \bf Solución pregunta 2
\end{center}

\bigskip

\noindent
Sea $\ell > 0$ un número entero, sea $\M$ el siguiente espacio de mensajes:
\begin{eqnarray*}
  \M &=& \{\varepsilon\} \cup \{0,1\} \cup \{0,1\}^2 \cup \{0,1\}^3 \cup \cdots \cup
  \{0,1\}^{\ell},
\end{eqnarray*}
donde $\varepsilon$ es la palabra vacía, y sea $\K
= \{0,1\}^{\ell+1}$. Defina un espacio de textos cifrados $\C$ que sea
subconjunto de $\{0,1\}^*$, y un esquema criptográfico
$(\Gen, \Enc, \Dec)$ sobre $\K$, $\M$ y $\C$ que sea perfectamente secreto.
\\
\\
{\bf Nota:} En la definición de $(\Gen, \Enc, \Dec)$ debe
suponer que $\Gen$ es la distribución uniforme sobre $\K$.

\bigskip

\noindent
{\bf Solución.} Dado $m \in \M$, definimos una función $f : \M \to
\{0,1\}^{\ell +1}$ de la siguiente forma. Para cada $m \in \M$, se
tiene que
\begin{eqnarray*}
  f(m) &=& m10^{\ell - |m|}.
\end{eqnarray*}
Vale decir, $f(m)$ es construido agregando a $m$ un símbolo 1 seguido
de $(\ell - |m|)$ símbolos 0. Nótese que $f$ es una función
inyectiva. Formalmente, dados $m_1, m_2 \in \M$ tales que $m_1 \neq
m_2$, tenemos que $f(m_1) \neq f(m_2)$ por los siguientes casos.
\begin{itemize}
\item Si $|m_1| = |m_2| = k$, entonces $f(m_1)$ difiere de $f(m_2)$ en
  alguno de los primeros $k$ símbolos, puesto que $m_1$ es prefijo de
  $f(m_1)$ y $m_2$ es prefijo de $f(m_2)$.

\item Si $|m_1| < |m_2|$, entonces $f(m_1)$ defiere de $f(m_2)$ ya que
  $f(m_2)$ tiene un símbolo 1 en la posición $|m_2| + 1$, mientras que
  $f(m_1)$ tiene un símbolo 0 en la posición $|m_2|+1$.
  
\item Si $|m_2| < |m_1|$, entonces se concluye que $f(m_1)$ defiere de
  $f(m_2)$ como en el caso anterior.
\end{itemize}
Como $f$ es una función inyectiva, denotamos como $f^{-1}$ a su inversa.

Sea $\C = \{ f(m) \mid m \in \M\}$, y defina las familias $\Enc$ y
$\Dec$ de la siguiente forma. Dado $k \in \K$, se tiene que:
\begin{itemize}
  \item para cada $m \in \M$: $\Enc_k(m) = f(m) \oplus k$, y 
  \item para cada $c \in \C$: $\Dec_k(m) = f^{-1}(c \oplus k)$.
\end{itemize}
Nótese que para cada $k \in \K$ y $m \in \M$:
\begin{eqnarray*}
  \Dec_k(\Enc_k(m)) &=& \Dec_k(f(m) \oplus k)\\
  &=& f^{-1}((f(m) \oplus k) \oplus k)\\
  &=& f^{-1}(f(m) \oplus (k \oplus k))\\
  &=& f^{-1}(f(m) \oplus 0^{\ell + 1})\\
  &=& f^{-1}(f(m))\\
  &=& m
\end{eqnarray*}
Por lo tanto el esquema criptográfico $(\Gen, \Enc, \Dec)$ está bien
definido, y para terminar la pregunta sólo tenemos que demostrar que
es perfectamente secreto. Vale decir, tenemos que demostrar que se
cumple la siguiente propiedad, dada una distribución de probabilidades
$\mathbb{D}$ para los mensajes en~$\M$:
$$
\forall m_0\in\mathcal{M}:\underset{
	\begin{array}{cc}
    	m\sim\mathbb{D}\\
    	k\sim\textit{Gen}
    \end{array}
}
{\Pr}[m=m_0\ |\ \textit{Enc}_k(m)=c_0]\quad = \quad 
\underset{m\sim \mathbb{D}}
{\Pr}[m=m_0].
$$
La demostración de que esta propiedad se cumple se puede hacer de
la misma forma que como se hizo en clases para el caso de OTP. En particular, se debe
considerar que $\Gen$ es la distribución uniforme sobre $\K$, y para
todo $m \in \M$ y $c \in \C$ existe un único $k$ tal que $\Enc_k(m) =
c$, por lo que se tiene que:
\begin{eqnarray*}
  \sum_{k\in\mathcal{K}\,:\,\Enc_k(m)=c} \Gen(k) &=& \frac{1}{2^{\ell+1}}.
\end{eqnarray*}


  
\end{document}
