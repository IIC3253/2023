\documentclass[11pt]{article}


\usepackage[utf8]{inputenc}
\usepackage{fullpage}
\usepackage{epsfig}
\usepackage{amsmath}
\usepackage{amssymb}
\usepackage{multicol}
\usepackage{color}
\usepackage{hyperref}
\usepackage{xcolor}
\usepackage{dirtree}
\usepackage{fontawesome}
\usepackage{tikz}


\usetikzlibrary{trees}


\newcommand{\comm}[1]{{\bf {\color{red} #1}}}
\newcommand{\Enc}{\textit{Enc}}
\newcommand{\Dec}{\textit{Dec}}
\newcommand{\Gen}{\textit{Gen}}

\newcommand{\M}{\mathcal{M}}
\newcommand{\C}{\mathcal{C}}
\newcommand{\K}{\mathcal{K}}

\begin{document}

\begin{center}
  \bf Criptografía y Seguridad Computacional - IIC3253\\
  \bf Tarea 1\\
  \bf Solución pregunta 3
\end{center}

\bigskip

\noindent
Considere el juego que define una PRP con una ronda sobre OTP. Demuestre que todo adversario tiene una probabilidad de ganar de exactamente 1/2.
\\
\\
\textbf{Bonus:} ¿Es cierto lo anterior si consideramos cualquier protocolo que satisface la propiedad de perfect secrecy? Demuestre o de un contraejemplo.

\bigskip

\noindent
{\bf Solución.} Trabajaremos sobre strings binarios de largo $n$, es decir $\mathcal{C}=\mathcal{M}=\mathcal{K}=\{0,1\}^n$. Comenzamos describiendo el juego para definir la notación. Al iniciar el juego, el Verificador elige $b=0$ o $b=1$. Si $b=0$, el Verificador también elige una llave $k$ y define $f:\mathcal{M}\rightarrow\mathcal{C}$ como $f(m)=\textit{Enc}_k(m)=m\oplus k$. De lo contrario, selecciona una permutación $\pi$ al azar y define $f(m)=\pi(m)$. Notar que tanto la elección de $b$ como la elección de $k$ siguen la distribución uniforme.

A continuación, el Adversario envía un mensaje $m$ al Verificador. El verificador responde con $r=f(m)$. Finalmente, el Adversario debe retornar 0 o 1, y gana si el valor retornado es igual a $b$.

Teniendo el setup anterior, para demostrar que la probabilidad de ganar de cualquier adversario es exactamente 1/2, pensaremos el problema de la siguiente forma: el adversario responde $b=0$ ó $b=1$ basándose la respuesta $r=f(m)$ que envía el adversario. La pregunta entonces es: ¿Cuál es la probabilidad de que $b$ haya sido cero o uno, dada la respuesta del verificador? Esto lo podemos calcular de la siguiente forma

$$\Pr_{b\sim \mathbb{U},\, k\sim \mathbb{U}}(b=0 \mid r=f(m))$$

Donde $\mathbb{U}$ representa la distribución uniforme en los respectivos espacios. Como las variables aleatorias seguirán siendo $b$ y $k$ las omitiremos. Aplicando teorema de Bayes obtenemos:

$$\Pr(b=0 \mid r=f(m))=
\frac{
  \Pr(r=f(m) \mid b=0)\cdot\Pr(b=0)
  }{
    \Pr(r=f(m))
}
$$

Tomamos la probabilidad del denominador y la dividimos en dos casos, $b=0$ y $b=1$ (como hicimos varias veces en clases), obteniendo que la expresión anterior es igual a:

$$\frac{
  \Pr(r=f(m)\mid b=0)\cdot\Pr(b=0)
  }{
    \Pr(r=f(m)\mid b=0)\cdot\Pr(b=0)+\Pr(r=f(m)\mid b=1)\cdot\Pr(b=1)
}
$$
\\

Como $\Pr(b=0)=\Pr(b=1)=1/2$, podemos simplificar lo anterior a
$$\frac{
  \Pr(r=f(m)\mid b=0)
  }{
    \Pr(r=f(m)\mid b=0)+\Pr(r=f(m)\mid b=1)
}
$$

Ahora calculamos por separado los dos términos que aparecen:

\begin{itemize}
  \item $\Pr(r=f(m)\mid b=0)$ es la probabilidad de que la respuesta del verificador haya sido $r$ dado que $f$ es la función de encriptación correspondiente a OTP. Es decir, la probabilidad de que $m\oplus k=r$. Dado que $k$ se elige de manera uniforme y existe exactamente una llave $k$ de las $2^n$ llaves posibles que cumple con $m\oplus k=r$, tenemos que $\Pr(r=f(m)\mid b=0)=1/2^n$.
  \item $\Pr(r=f(m)\mid b=1)$ es la probabilidad de que la respuesta del verificador haya sido $r$ dado que $f$ es una permutación aleatoria. Esta probabilidad la hemos calculado antes en clases, y es simplemente la probabilidad de que al sacar un string de $\{0,1\}^n$ al azar obtengamos exactamente $r$, es decir $1/2^n$.
\end{itemize}

Tenemos entonces
$$\frac{
  \Pr(r=f(m)\mid b=0)
  }{
    \Pr(r=f(m)\mid b=0)+\Pr(r=f(m)\mid b=1)
}=\frac{
  \frac{1}{2^n}
}{
  \frac{1}{2^n}+\frac{1}{2^n}
}=\frac{1}{2}
$$

Recapitulando, lo que hicimos fue demostrar que 

$$\Pr_{b\sim \mathbb{U},\, k\sim \mathbb{U}}(b=0\mid r=f(m))= \frac{1}{2}$$

Es decir, conociendo la respuesta que dio el verificador, para el adversario la probabilidad de que $b$ haya sido $0$ es exactamente 1/2. Obviamente esto implica que la probabilidad de que $b$ haya sido $1$ es también 1/2. Por lo tanto, si el adversario responde $0$ o responde $1$, su probabilidad de ganar es exactamente 1/2.

\end{document}
