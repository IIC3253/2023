\documentclass[11pt]{article}


\usepackage[utf8]{inputenc}
\usepackage{fullpage}
\usepackage{epsfig}
\usepackage{amsmath}
\usepackage{amssymb}
\usepackage{multicol}
\usepackage{color}
\usepackage{hyperref}
\usepackage{xcolor}
\usepackage{dirtree}
\usepackage{fontawesome}
\usepackage{tikz}


\usetikzlibrary{trees}


\newcommand{\comm}[1]{{\bf {\color{red} #1}}}
\newcommand{\Enc}{\textit{Enc}}
\newcommand{\Dec}{\textit{Dec}}
\newcommand{\Gen}{\textit{Gen}}


\begin{document}

\begin{center}
  \bf Criptografía y Seguridad Computacional - IIC3253\\
  \bf Tarea 1\\
  \bf Solución pregunta 4
\end{center}


\bigskip

\noindent
En ayudantía fue demostrado que si una función de hash es resistente a
colisiones, entonces esta función debe ser resistente a preimagen. En
esta pregunta usted debe demostrar que la implicación inversa no es
cierta. Vale decir, suponiendo que existe una función de hash que es
resistente a preimagen, demuestre que existe una función de hash
$(\Gen, h)$ que es resistente a preimagen y no es resistente
a colisiones.


\bigskip

\noindent
{\bf Solución.}  Suponga que $(\Gen, h')$ es una función de hash
resistente a preimagen. En particular, para cada $n \geq 0$, si
$\Gen(1^n) = s$, entonces $(h')^s : \{0,1\}^* \to \{0,1\}^{\ell(n)}$
donde $\ell(n)$ es un polinomio fijo. A partir de esta función,
definimos una función de hash $(\Gen, h)$ de la siguiente
forma. Suponiendo que $n \geq 0$ y $\Gen(1^n) = s$, para cada $m \in
\{0,1\}^*$ se tiene que:
\begin{eqnarray*}
h^s(m) &=&
\begin{cases}
  (h')^s(\varepsilon) & \text{si } m = \varepsilon\\
  (h')^s(u) & \text{si } m = uv \text{ con } |v| = 1
\end{cases}
\end{eqnarray*}
Vamos a demostrar que $(\Gen, h)$ es resistente a preimagen y no es
resistente a colisiones.

Suponga primero que $(\Gen, h)$ no es resistente a preimagen, de lo
cual esperamos llegar a una contradicción.  Dado que $(\Gen, h)$ no es
resistente a preimagen, existe un algoritmo aleatorizado ${\cal A}$ de
tiempo polinomial que gana el siguiente juego con una probabilidad no
despreciable. Dado $n \geq 0$, se ejecutan los siguientes pasos:
\begin{enumerate}
\item El verificador genera $s = \Gen(1^n)$ y un hash
  $x \in \{0,1\}^{\ell(n)}$
\item El adversario elige $m \in \{0,1\}^*$ o $m = \bot$
\item El adversario gana el juego si alguna de las siguientes condiciones se cumple:
\begin{itemize}
\item $m \in \{0,1\}^{*}$ y $h^s(m) = x$ 
\item $m = \bot$ y  no existe $m' \in \{0,1\}^*$ tal que $h^s(m') = x$
\end{itemize}
En caso contrario, el adversario pierde.
\end{enumerate}
A partir del algoritmo ${\cal A}$, definimos un algoritmo aleatorizado
${\cal A}'$ de la siguiente forma. Dado $n \geq 0$ y $x \in
\{0,1\}^{\ell(n)}$, el algoritmo ${\cal A}'$ se pone en el papel del
verificador en el juego anterior, y le pide a ${\cal A}$ una preimagen
para $x$. Si ${\cal A}$ responde con $m = \bot$ o $m = \varepsilon$,
entonces ${\cal A}'$ responde con el mismo string $m$ como una
preimagen para $x$ bajo la función $(\Gen, h')$. Si ${\cal A}$
responde con $m = uv$ con $m \in \{0,1\}^*$ y $|v| = 1$, entonces
entonces ${\cal A}'$ responde con $u$ como una preimagen para $x$ bajo
la función $(\Gen, h')$. Tenemos que ${\cal A'}$ es un algoritmo
aleatorizado de tiempo polinomial ya que ${\cal A}$ es un algoritmo
aleatorizado de tiempo polinomial. Además, ${\cal A}'$ genera una
preimagen de $x$ con la misma probabilidad que ${\cal A}$, puesto que
por definición de $h$ tenemos que:
\begin{itemize}
\item si $m = \bot$ y no existe $m'$ tal que $h^s(m') = x$, entonces
  no existe $m'$ tal que $(h')^s(m') = x$;

\item si $m = \varepsilon$ y $h^s(m) = x$, entonces $(h')^s(\varepsilon) =
  h^s(\varepsilon) = x$; y

\item si $m = uv$, con $m \in \{0,1\}^*$ y $|v| = 1$, y $h^s(m) = x$,
  entonces $(h')^s(u) = h^s(uv) = x$.
\end{itemize}
La existencia del algoritmo ${\cal A}'$ nos muestra que la
función de hash $(\Gen, h')$ no es resistente a preimagen, lo cual
contradice nuestro supuesto inicial.

Para demostrar que $(\Gen, h)$ no es resistente a colisiones, nos
basta considerar que si $n \geq 0$ y $\Gen(1^n) = s$, entonces $h^s(0)
= h^s(1) = (h')^s(\varepsilon)$. Esto concluye el ejercicio.
\end{document}
