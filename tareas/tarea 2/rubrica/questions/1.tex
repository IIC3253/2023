%!TEX root = ../main/main.tex

En esta pregunta usted va a implementar y demostrar la corrección de un
esquema criptográfico que utiliza claves más simples que las de RSA, y
que además es aleatorizado, produciendo con una alta probabilidad cifrados distintos si un
mensaje es encriptado más de una vez.

Para definir este esquema, necesitamos introducir un poco de
notación. Dado un número natural $n$, sea $\#\text{Bit}(n)$ el número
de bits en la representación binaria de $n$. Además, dados dos números
naturales $n$, $m$ tales que $m > 0$, sea
  \begin{eqnarray*}
\dv(n,m) &=& \bigg\lfloor \frac{n-1}{m} \bigg\rfloor.
  \end{eqnarray*}
Entonces, la clave pública $P_A$ y la clave
secreta $S_A$ de un usuario $A$ son generadas de la siguiente forma.
\begin{enumerate}
\item[(a)] Genere dos números primos distintos $P$ y $Q$ tales que $P \geq
  3$, $Q \geq 3$ y $\bbit(P) = \bbit(Q)$.  Sea $N = P \cdot Q$ y
  $\phi(N) = (P-1) \cdot (Q-1)$.

\item[(b)] Defina $P_A = N$ y $S_A = \phi(N)$.
\end{enumerate}
La función de cifrado $\Enc_{P_A}$ es definida de la siguiente
forma. Dado un mensaje $m \in \{0, \ldots, N-1\}$, se genera al azar un número $r
\in \{1, \ldots, N-1\}$ tal que $\MCD(r, N) = 1$, y se construye
  \begin{eqnarray*}
    \Enc_{P_A}(m) &=& ((N+1)^m \cdot r^N) \!\! \mod N^2
  \end{eqnarray*}
La función de descifrado $\Dec_{S_A}$ es definida de la siguiente
forma. Sea $B \in \{0, \ldots, N-1\}$ el inverso de $\phi(N)$ en módulo
$N$, vale decir, $B$ satisface la condición
\begin{eqnarray*}
  \phi(N) \cdot B \equiv 1  \mod N
\end{eqnarray*}
Entonces dado un texto cifrado $c \in \{0, \ldots, N^2-1\}$, se define 
  \begin{eqnarray*}
    \Dec_{S_A}(c) &=&   \big[\dv(c^{\phi(N)} \!\!\!\mod N^2,\, N)  \cdot B \big] \!\!\mod N
  \end{eqnarray*}
  Responda las siguientes preguntas, en las cuales va a implementar el esquema criptográfico y va a demostrar que es correcto.
    \begin{enumerate}
\item[(a)] Implemente el esquema criptográfico definido en esta pregunta
  completando el Jupyter notebook
  \href{https://github.com/IIC3253/2023/blob/main/tareas/tarea\%202/enunciado/questions/p1/pregunta1_a.ipynb}{\texttt{pregunta1\_a.ipynb}}. Para
  que su pregunta sea considerada correcta, su notebook deberá correr
  de principio a fin habiendo completado los métodos marcadas con
  \texttt{\#\#\#\#\# POR COMPLETAR}. Las entradas y salidas de estos
  métodos no pueden ser modificadas, pero sí puede agregar métodos
  adicionales si los considera necesarios. Se
  evaluará con un programa externo la implementación de sus clases
  \texttt{Receiver} y \texttt{Sender}.
  
  \item[(b)] Demuestre que $\MCD(N, \phi(N)) = 1$. Nótese que de esto
    se deduce la existencia del número $B$, que es el inverso
    de $\phi(N)$ en módulo $N$.

\item[(c)] Dado     $m \in \{0, \ldots, N-1\}$, demuestre que:
  \begin{eqnarray*}
    \Dec_{S_A}(\Enc_{P_A}(m)) &=& m
  \end{eqnarray*}
  \end{enumerate}


\medskip

\paragraph{Corrección.}
Está pregunta será evaluada de la siguiente forma.
\begin{itemize}
  \item[(a)] Se realizaran cuatros tests, cada uno de los cuales
  incluye una llamada a la clase \texttt{Receiver} para crear una
  clave pública con un cierto número de bits, una llamada a la
  clase \texttt{Sender} para establecer como clave pública la generada
  por \texttt{Receiver}, y dos llamadas para cifrar un mismo mensaje
  y descifrarlo. Este tipo de tests es el mismo que se utilizó en el
  Jupyter
  notebook \href{https://github.com/IIC3253/2023/blob/main/tareas/tarea\%202/enunciado/questions/p1/pregunta1_a.ipynb}{\texttt{pregunta1\_a.ipynb}}
  que usted debió completar. Por cada test que sea realizado
  correctamente, usted recibirá 0.5 puntos.

  \item[(b)] La asignación de puntaje en esta pregunta es la siguiente.
\begin{itemize}
    \item{[1 punto]} Tiene una idea correcta sobre cómo demostrar que
    $\MCD(N, \phi(N)) = 1$, pero la formalización de la demostración
    está incompleta o incorrecta.

    \item{[2 puntos]} Tiene una idea correcta sobre cómo demostrar que
    $\MCD(N, \phi(N)) = 1$, y esta idea está correctamente
    formalizada.
\end{itemize}

  \item[(c)] La asignación de puntaje en esta pregunta es la siguiente.
\begin{itemize}
    \item{[1 punto]} Tiene una idea correcta sobre cómo demostrar que
    $\Dec_{S_A}(\Enc_{P_A}(m)) = m$, pero la formalización de la
    demostración está incompleta o incorrecta.

   \item{[2 puntos]} Tiene una idea correcta sobre cómo demostrar que
   $\Dec_{S_A}(\Enc_{P_A}(m)) = m$, y esta idea está correctamente
   formalizada.
\end{itemize}
\end{itemize}

\medskip
