%!TEX root = ../main/main.tex

\pagebreak

En esta pregunta deberá escribir un programa que verifique la autenticidad de un Json Web Token (JWT) en sus variantes HS256 y RS256. Para esto, deberá escribir un Jupyter notebook siguiendo las instrucciones explicadas arriba que al menos defina la siguiente función:

\begin{python}
def validate_jwt(jwt: str, key: str) -> bool:
    """
    Arguments:
      jwt: a well-formed Json Web Token
      key: the key to verify the validity of the jwt
    Returns:
      valid: is the jwt is valid w.r.t the provided key?
    """
\end{python}
Su notebook podrá definir funciones auxiliares que ayuden a simplificar la lectura. Deberá estar explicado y ser fácil de seguir para una persona que entiende los contenidos del curso.

\textbf{Importante}: Su notebook sólo podrá importar las siguientes librerías externas
\begin{python}
from hashlib import sha256
from base64 import urlsafe_b64decode, urlsafe_b64encode
\end{python}


Para validar un JWT que usa HS256, el parámetro \texttt{key} es un string que representa la llave a usar en HMAC-SHA256. Por ejemplo el siguiente JWT es válido con la llave \texttt{IIC3253}.

\begin{tabular}{l}
  \texttt{eyJhbGciOiJIUzI1NiIsInR5cCI6IkpXVCJ9.eyJjbGFzcyI6IkNyaXB0b2dy}\\
  \texttt{YWbDrWEgeSBTZWd1cmlkYWQgQ29tcHV0YWNpb25hbCIsInVuaXZlcnNpdHkiO}\\
  \texttt{iJQVUMgQ2hpbGUifQ.Cn1AACqINaUTbAJuh\_V4lBcr9X4dRp8FUX9sGDkX-Ss}
\end{tabular}

Para un JWT que usa RS256, el parámetro \texttt{key} es una llave pública RSA en formato \href{https://www.rfc-editor.org/rfc/rfc8017}{\texttt{PKCS\#1}} para validar la firma RSA contenida en el JWT. Por ejemplo el JWT

\begin{verbatim}
  eyJhbGciOiJSUzI1NiIsInR5cCI6IkpXVCJ9.eyJuYW1lIjoiQ3JpcHRvZ3JhZsOt
  YSB5IFNlZ3VyaWRhZCBDb21wdXRhY2lvbmFsIiwiY291cnNlIjoiSUlDMzI1MyIsI
  nVuaXZlcnNpdHkiOiJQVUMgQ2hpbGUifQ.CmoMizX-_E2Ugd7-JDThCfrXTJbg38W
  Lal3HipmnA8oAUh1yG9IU1n_klJkmPIT3knxrmrJMXxh6gTCOylLQfKSQI7pHsYUr
  -yOd5gL7XpnT3stvOtYDO383cBnrL5X8EV01lUxJJenYG5Qr4uVG7Msg-4fUJbTqT
  R2t0Jx2UQ2pfi_jxgfg6lAjSLK9TygntJJ-eJV0Q8IipVYnqtCxBS-OIXekalyjpB
  Hksf_ibiJtPrMJI3Kvyj3dwrETth8c4yg2IIh22uoJHrJArNk3xPfeSsasZT0ixfM
  E8Mlnkd4HwpbcNZZl-FpsBPbPWHKynZXptq8uS65PxKmTggg8kA
\end{verbatim}

es válido con la llave pública

\begin{verbatim}
  -----BEGIN PUBLIC KEY-----
  MIIBIjANBgkqhkiG9w0BAQEFAAOCAQ8AMIIBCgKCAQEAu1SU1LfVLPHCozMxH2Mo
  4lgOEePzNm0tRgeLezV6ffAt0gunVTLw7onLRnrq0/IzW7yWR7QkrmBL7jTKEn5u
  +qKhbwKfBstIs+bMY2Zkp18gnTxKLxoS2tFczGkPLPgizskuemMghRniWaoLcyeh
  kd3qqGElvW/VDL5AaWTg0nLVkjRo9z+40RQzuVaE8AkAFmxZzow3x+VJYKdjykkJ
  0iT9wCS0DRTXu269V264Vf/3jvredZiKRkgwlL9xNAwxXFg0x/XFw005UWVRIkdg
  cKWTjpBP2dPwVZ4WWC+9aGVd+Gyn1o0CLelf4rEjGoXbAAEgAqeGUxrcIlbjXfbc
  mwIDAQAB
  -----END PUBLIC KEY----- 
\end{verbatim}

\medskip

\paragraph{Corrección.}
Para evaluar esta pregunta se utilizan los 20 pares JWT/llave que se muestran en el Anexo~\ref{appendix:jwts-keys}. Se evaluarán 40 pares: los 20 que aparecen en el anexo, que son válidos, y 20 pares inválidos que se obtendrán aleatoriamente sacando tokens y llaves del mismo anexo.

El puntaje obtenido se calcula como $6\cdot r\cdot (0.7 + 0.3\cdot d)$, donde $r$ (ratio) es la proporción de JWTs correctamente validados/rechazados y $d$ (desarrollo) se calcula de acuerdo a los puntajes que se muestran más abajo. Por ejemplo, si de los 40 pares llave/JWT a evaluar se consigue validar correctamente 30 y $d$ es $0.5$, entonces el puntaje obtenido en esta pregunta será $6\cdot 30/40 \cdot (0.7 + 0.3\cdot 0.5) = 3.825$.

Cálculo del valor $d$:
\begin{itemize}
  \item{[0]} Entrega un notebook vació o que no aporta información concreta respecto de cómo se valida un JWT en las modalidades RS256 y HS256.
  \item{[0.25]} El notebook explica parcialmente cómo programar lo necesario para validar JWTs en una de las modalidades y no explica prácticamente nada de la otra.
  \item{[0.5]} El notebook explica de forma concreta y correcta cómo programar lo necesario para validar JWTs en una de las modalidades y no explica prácticamente nada de la otra.
  \item{[0.75]} El notebook explica de forma concreta y correcta cómo programar lo necesario para validar JWTs en una de las modalidades, y parcialmente cómo hacerlo para validar JWTs en la otra.
  \item{[1]} El notebook describe de forma concreta y correcta lo que se programó, que es exactamente lo que se necesita para validar JWTs en las modalidades RS256 y HS256.
\end{itemize}

\medskip

