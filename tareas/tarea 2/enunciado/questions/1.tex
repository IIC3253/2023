%!TEX root = ../main/main.tex

\newcommand{\bbit}{\#\text{Bit}}
\newcommand{\dv}{\text{Div}}

En esta pregunta used va a implementar y demostrar la corrección de un
esquema criptográfico que utiliza claves más simples que las de RSA, y
que además es aleatorizado, produciendo con una alta probabilidad cifrados distintos si un
mensaje es encriptado más de una vez.

Para definir este esquema, necesitamos introducir un poco de
notación. Dado un número natural $n$, sea $\#\text{Bit}(n)$ el número
de bits en la representation binaria de $n$. Además, dados dos números
naturales $n$, $m$ tales que $m > 0$, sea
  \begin{eqnarray*}
\dv(n,m) &=& \bigg\lfloor \frac{n-1}{m} \bigg\rfloor.
  \end{eqnarray*}
Entonces, la clave pública $P_A$ y la clave
secreta $S_A$ de un usuario $A$ son generadas de la siguiente forma.
\begin{enumerate}
\item Genere dos números primos distintos $P$ y $Q$ tales que $P \geq
  3$, $Q \geq 3$ y $\bbit(P) = \bbit(Q)$.  Sea $N = P \cdot Q$ y
  $\phi(N) = (P-1) \cdot (Q-1)$.

\item Defina $P_A = N$ y $S_A = \phi(N)$.
\end{enumerate}
La función de cifrado $\Enc_{P_A}$ es definida de la siguiente
forma. Dado un mensaje $m \in \{0, \ldots, N-1\}$, se genera al azar un número $r
\in \{1, \ldots, N-1\}$ tal que $\MCD(r, N) = 1$, y se construye
  \begin{eqnarray*}
    \Enc_{P_A}(m) &=& ((N+1)^m \cdot r^N) \!\! \mod N^2
  \end{eqnarray*}
La función de descifrado $\Dec_{S_A}$ es definida de la siguiente
forma. Sea $B \in \{0, \ldots, N-1\}$ el inverso de $\phi(N)$ en módulo
$N$, vale decir, $B$ satisface la condición
\begin{eqnarray*}
  \phi(N) \cdot B \equiv 1  \mod N
\end{eqnarray*}
Entonces dado un texto cifrado $c \in \{0, \ldots, N^2-1\}$, se define 
  \begin{eqnarray*}
    \Dec_{S_A}(c) &=&   \big[\dv(c^{\phi(N)} \!\!\!\mod N^2,\, N)  \cdot B \big] \!\!\mod N
  \end{eqnarray*}
  Responda las siguientes preguntas, en las cuales va a implementar el esquema criptográfico y va a demostrar que es correcto.
    \begin{enumerate}
\item Implemente el esquema criptográfico definido en esta pregunta completando el Jupyter
  notebook
  \href{https://github.com/IIC3253/2023/blob/main/tareas/tarea\%202/enunciado/questions/p1/pregunta1_a.ipynb}{\texttt{pregunta1\_a.ipynb}}. Para
  que su pregunta sea considerada correcta, su notebook deberá correr
  de principio a fin habiendo modificado exclusivamente las funciones
  marcadas con \texttt{\#\#\#\#\# POR COMPLETAR}. En particular, se
  evaluará con un programa externo la implementación de sus clases
  \texttt{Receiver} y \texttt{Sender}.
  
  \item Demuestre que $\MCD(N, \phi(N)) = 1$. Nótese que de esto
    se deduce la existencia del número $B$, que es el inverso
    de $\phi(N)$ en módulo $N$.

\item Dado     $m \in \{0, \ldots, N-1\}$, demuestre que:
  \begin{eqnarray*}
    \Dec_{S_A}(\Enc_{P_A}(m)) &=& m
  \end{eqnarray*}
  \end{enumerate}
