\documentclass[11pt]{article}


\usepackage[utf8]{inputenc}
\usepackage{fullpage}
\usepackage{epsfig}
\usepackage{amsmath}
\usepackage{amssymb}
\usepackage{multicol}
\usepackage{color}
\usepackage{hyperref}
\usepackage{xcolor}
\usepackage{dirtree}
\usepackage{fontawesome}
\usepackage{tikz}


\usetikzlibrary{trees}


\newcommand{\comm}[1]{{\bf {\color{red} #1}}}
\newcommand{\Enc}{\textit{Enc}}
\newcommand{\Dec}{\textit{Dec}}
\newcommand{\Gen}{\textit{Gen}}

\newcommand{\M}{\mathcal{M}}
\newcommand{\C}{\mathcal{C}}
\newcommand{\K}{\mathcal{K}}

\newcommand{\MCD}{\textit{MCD}}

\begin{document}

\begin{center}
  \bf Criptografía y Seguridad Computacional - IIC3253\\
  \bf Tarea 3\\
  \bf Solución pregunta 1
\end{center}

\bigskip

\noindent

\newcommand{\bbit}{\#\text{Bit}}
\newcommand{\dv}{\text{Div}}

El objetivo de esta pregunta es que usted implemente el protocolo
criptográfico ElGamal y las firmas de Schnorr sobre grupos
arbitrarios, y en particular que lo utilice sobre grupos generados por
curvas elípticas. Para hacer esto, deberá completar el Jupyter
notebook
\href{https://github.com/IIC3253/2023/blob/main/tareas/tarea\%203/enunciado/questions/p1/pregunta1.ipynb}{\texttt{pregunta1.ipynb}},
en el cual primero deberá implementar el protocolo ElGamal y las
firmas de Schnorr sobre una representación general de grupos, para
luego probar su implementación sobre los grupos $\mathbb{Z}_p^*$
estudiados en clases, y finalmente probar su implementación sobre grupos
generados por curvas elípticas como son definidos en el siguiente
libro:
\begin{itemize}
  \item Jonathan Katz y Yehuda Lindell. {\em Introduction to Modern Cryptography}. Chapman and Hall/CRC, tercera edición, 2020.
\end{itemize}
Para que su pregunta sea considerada correcta, su notebook deberá
correr de principio a fin habiendo modificado exclusivamente las
clases y funciones marcadas con \texttt{\#\#\#\#\# POR COMPLETAR}. En
particular, se evaluará con un programa externo la implementación de
sus clases \texttt{SecretKeyHolder}, \texttt{PublicKeyHolder} y
\texttt{EllipticCurve}.



\bigskip

\noindent
    {\bf Solución.}
    Una solución de la tarea está en el Jupyter Notebook
  \href{https://github.com/IIC3253/2023/blob/main/tareas/tarea\%203/soluciones/p1/sol_tests_p1.ipynb}{\texttt{sol\_tests\_p1.ipynb}}.
  
\end{document}
